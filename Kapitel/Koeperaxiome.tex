\documentclass[../AbiMappe_Mathe.tex]{subfiles}

\begin{document}

\theoremstyle{nonumberplain}
\newtheorem{mytheo}{Remark}
\newframedtheorem{fmytheo}{}

\section{Körper-Axiome}
Diese Axiome muss eine Menge erfuellen um als Koeper bezeichnet werden zu duerfen.

\begin{fmytheo}
\vspace{-0.5 cm}
\begin{align*}
K1.\hspace{0.3 em} &\textbf{Kommutativgesetze.}& \hspace{-1.0 cm} &a+b=b+a,\; a*b=b*a.\\
K2.\hspace{0.3 em} &\textbf{Assoziativgesetze.}& \hspace{-1.0 cm} &a+(b+c)=(a+b)+c,\; a(bc)=(ab).\\
K3. \hspace{0.3 em} &\textbf{Distributivgesetz.}& \hspace{-1.0 cm} &a(b+c)=ab+ac.\\
K4.  \hspace{0.3 em} &\textbf{Existenz neutraler Elemente.}& \hspace{-1.0 cm} &a+0=a,\; a*1=a.\\
K5.  \hspace{0.3 em} &\textbf{Existenz inverser Elemente.}& \hspace{-1.0 cm} &a+(-a)=0 ,\; a*a^{-1}=1,\; 1 \neq 0.
\end{align*}
\end{fmytheo}

\section{Ordnungs-Axiome}
Diese Axiome muss eine Koerper erfuellen um als geordneter Koerper bezeichnet werden zu duerfen.
\end{document}



% \; \forall a \in \mathds{R}

% \begin{align*}
% K1.\hspace{0.3 em} &\textbf{Kommutativgesetze.}& \hspace{-1.5 cm} &a+b=b+a,\; a*b=b*a\\
% K2.\hspace{0.3 em} &\textbf{Assoziativgesetze.}& \hspace{-1.5 cm} &a+(b+c)=(a+b)+c,\; a(bc)=(ab)
% \end{align*}

% \newmdtheoremenv{theo}{}
% \begin{theo}
% \begin{align*}
% K1.\hspace{0.3 em} &\textbf{Kommutativgesetze.}& \hspace{-1.5 cm} &a+b=b+a,\; a*b=b*a\\
% K2.\hspace{0.3 em} &\textbf{Assoziativgesetze.}& \hspace{-1.5 cm} &a+(b+c)=(a+b)+c,\; a(bc)=(ab)
% \end{align*}

% \end{theo}