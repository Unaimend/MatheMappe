\documentclass[../AbiMappe_Mathe.tex]{subfiles}

\begin{document}

\theoremstyle{nonumberplain}

\newframedtheorem{fmytheo1}{}

\section{Peano-Axiome}
Die Peano-Axiome dienen zur Definition der natuerliche Zahlen.

\begin{fmytheo1}
\vspace{-0.5 cm}
\begin{align*}
P1.\hspace{0.3 em} &\textbf{0}\;ist\; eine\; ganze\; Zahl.\\
P2.\hspace{0.3 em} &Wenn \; \textbf{n}\;\text{eine ganze Zahl ist, ist \textbf{n++} es auch.}\\
P3. \hspace{0.3 em} &\text{Es existiert kein \textbf{n} fuer das gilt \textbf{n++ = 0}} \\
P4.  \hspace{0.3 em} &\text{Wenn \textbf{n'} = \textbf{m'} folgt daraus  \textbf{n} = \textbf{m}.} \\
P5.  \hspace{0.3 em} &\text{Enthält \textbf{X} die \textbf{0} und mit jeder natürlichen Zahl \textbf{n}}\\
&\text{auch deren Nachfolger \textbf{n'}, so bilden die \textbf{natuerlichen} \textbf{Zahlen} eine Teilmenge von \textbf{X}.}
\end{align*}
\end{fmytheo1}

\end{document}
