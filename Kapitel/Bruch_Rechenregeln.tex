\documentclass[../AbiMappe_Mathe.tex]{subfiles}

\begin{document}

\theoremstyle{nonumberplain}
\newtheorem{mytheo3}{Remark}
\newframedtheorem{fmytheo3}{}
\section{Bruchrechnung}
Diesen Rechengesetzen unterliegt die Bruchrechnung.


\subsection{Erweitern}
Es seien a,b,n $\in$ $\mathds{Z}$ mit b,n $\neq$ 0. Dann gillt.\\\\\noindent\hspace*{10mm}%
$
	\frac{a}{b}=\frac{n*a}{n*b}
$\\\\
Beweis. Es gilt $\frac{a}{b}=\frac{n*a}{n*b}$ $\Leftrightarrow$  $\frac{a*(n*b)}{b}=\frac{n*a}{1}$  
$\Leftrightarrow$ $\frac{a*(n*b)}{1}=\frac{n*a*b}{1}$
$\Leftrightarrow$  $\frac{a*n*b}{1}=\frac{a*n*b}{1}$.
\begin{flushright}$\square$\end{flushright}


\subsection{Kuerzen}


\subsection{Addition}
Brueche mit gleichem Nenner addiert indem man die Zaehler/Nenner addiert.
Es seien a,b,c $\in$ $\mathds{Z}$ mit b $\neq$ 0, dann gilt\\\\\noindent\hspace*{10mm}%
$
	\frac{a}{b}+\frac{c}{b}=\frac{a+c}{b},
$\\\\
falls die Brueche nicht den gleichen Nenner besitzen muss man sie auf einen gemeinsamen Hauptnenner bringen.\\\\
Beweis. Es gilt $\frac{a}{b}+\frac{c}{d}=\frac{ad+bc}{bd}$ da $\frac{a}{b}+\frac{c}{d}$
$\Leftrightarrow$ $\frac{a*d}{b*d}+\frac{c*b}{d*b}$
$\Leftrightarrow$ $\frac{a*d+c*b}{b*d}$.
\begin{flushright}$\square$\end{flushright}



\subsection{Subtraktion}
Brueche mit gleichem Nenner subtrahiert indem man die Zaehler/Nenner subtrahiert.
Es seien a,b$\in$ $\mathds{Z}$, c,d $\in$  $\mathds{Z}$ $\setminus$\{0\}, dann gilt\\\\\noindent\hspace*{10mm}%
$
	\frac{a}{b}-\frac{c}{b}=\frac{a-c}{b},
$\\\\
falls die Brueche nicht den gleichen Nenner besitzen muss man sie auf einen gemeinsamen Hauptnenner bringen.
Es seien a,b$\in$ $\mathds{Z}$, c,d $\in$  $\mathds{Z}$ $\setminus$\{0\}, dann gilt\\\\\noindent\hspace*{10mm}%
$
	\frac{a}{b}-\frac{c}{d}=\frac{ad-bc}{bd}
$\\\\
Beweis. Es gilt $\frac{a}{b}-\frac{c}{d}=\frac{ad-bc}{bd}$ da $\frac{a}{b}-\frac{c}{d}$
$\Leftrightarrow$ $\frac{a*d}{b*d}-\frac{c*b}{d*b}$
$\Leftrightarrow$ $\frac{a*d-c*b}{b*d}$.
\begin{flushright}$\square$\end{flushright}




\subsection{Multiplikation}
Brueche werden multipliziert indem man Zaehler * Zaehler und Nenner * Nenner rechnet.
Es seien a,c$\in$ $\mathds{Z}$, b,d $\in$ $\mathds{Z}$ $\setminus$\{0\}, dann gilt\\\\\noindent\hspace*{10mm}% 
$
	\frac{a}{b}*\frac{c}{d}=\frac{a*c}{b*d}.
$
\subsection{Division}
Um die Division durch Brueche zu ermoegliche muss man zuerst ihr multipikatives inverses bestimment, also das r
welches die Gleichung $\frac{a}{b}*r = r*\frac{a}{b}=1$ erfuellt. Nach der Regel zur Multiplikation gilt\\\\\noindent\hspace*{10mm}% 
$
	\frac{a}{b} * \frac{b}{a} = \frac{a*b}{b*a} = 1
$
\\also gilt \\\\\noindent\hspace*{10mm}%
$\frac{a}{b} * \frac{1}{\frac{a}{b}}= \frac{a}{b} * (\frac{a}{b})^{-1}=1$\\
das heist. Es seien a$\in$ $\mathds{Z}$, b,c,d $\in$ $\mathds{Z}$ $\setminus$\{0\}, dann gilt\\\\\noindent\hspace*{10mm}% 
$
	\frac{a}{b} : \frac{c}{d} = \frac{a}{b} * (\frac{c}{d})^{-1} = \frac{a}{b} * \frac{d}{c} = \frac{a*d}{b*c}.
$
\subsection{Zusammenfassung}
Es seien a,c$\in$ $\mathds{Z}$, b,d $\in$ $\mathds{Z}$ $\setminus$\{0\}.\\
\noindent\rule{4cm}{0.4pt}\\
Addition von Bruechen	 \hspace{2cm}$\frac{a}{b}+\frac{c}{d}=\frac{ab+bc}{bd}$\\
\noindent\rule{4cm}{0.4pt}\\
Subktraktion von Bruechen \hspace{2cm}	$\frac{a}{b}-\frac{c}{d}=\frac{ab-bc}{bd}$\\
\noindent\rule{4cm}{0.4pt}\\
Multiplikation von Bruechen\\
\noindent\rule{4cm}{0.4pt}\\
Teilen durch einen Bruch\\
\noindent\rule{4cm}{0.4pt}\\
Division von Bruechen\\
\noindent\rule{4cm}{0.4pt}\\
\end{document}



% \; \forall a \in \mathds{R}

% \begin{align*}
% K1.\hspace{0.3 em} &\textbf{Kommutativgesetze.}& \hspace{-1.5 cm} &a+b=b+a,\; a*b=b*a\\
% K2.\hspace{0.3 em} &\textbf{Assoziativgesetze.}& \hspace{-1.5 cm} &a+(b+c)=(a+b)+c,\; a(bc)=(ab)
% \end{align*}

% \newmdtheoremenv{theo}{}
% \begin{theo}
% \begin{align*}
% K1.\hspace{0.3 em} &\textbf{Kommutativgesetze.}& \hspace{-1.5 cm} &a+b=b+a,\; a*b=b*a\\
% K2.\hspace{0.3 em} &\textbf{Assoziativgesetze.}& \hspace{-1.5 cm} &a+(b+c)=(a+b)+c,\; a(bc)=(ab)
% \end{align*}

% \end{theo}