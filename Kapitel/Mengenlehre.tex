\documentclass[../AbiMappe_Mathe.tex]{subfiles}

\begin{document}

\theoremstyle{nonumberplain}

\newframedtheorem{fmytheo2}{}


\section{Naive Mengenlehre}
Eine Menge ist eine Zusammenfassung von wohlbestimmten Objekten zu einem Ganzen.
Diese Objekte heissen Elemente.\\\\
\textbf{Elementbeziehung}\\
Sei $M$ eine beliebiege nichtleere Menge dann bedeutet, $x \in M$, das wir ein beliebiges $x$ der Menge $M$ auswaehlen.

\subsection{Angabe von Mengen}
\textbf{Aufzaehlung:}\\
Eine endliche Menge kann durch aufzaehlung all ihrer Elemente angegeben werde z.B. stellt\\ $M=\{1,2,3,4,5\}$, die Menge aller natuerliche Zahlen $<6$ dar.\\\\
\textbf{Bildungsgesetz:}\\
Eine unedliche Menge kann mit Hilfe eines Bildungsgesetzes angegeben werden z.B. \\$M=\{1,2,3,\dots\}=\mathbb{N}$\\\\
\textbf{Eigenschaft:}\\
Sei $M$ eine Menge und $E$ eine Eingenschaft die alle Elemente der Menge $M$ entweder besitzen oder nicht
\subsection{Mengenbeziehungen}

\subsubsection{Teilmenge:}
\textbf{Definition:} Sei M eine Menge. Dann heisst eine weitere Menge N Teilmenge von M wenn gilt:
\begin{align*}
x \in N \Rightarrow x \in M
\end{align*}
\textbf{Notation:} 
\begin{align*}
N \subseteq M
\end{align*}


\subsection{Potenzmenge}
\textbf{Defintion:} Sei $M$ eine Menge, dann nennt die Menge all ihrer Teilmengen $U$ Potenzmenge der Menge $M$.
\begin{align*}
\mathcal{P}(M) := \{U|U \subseteq M  \}
\end{align*}
\\\textbf{Beispiel:} 
\begin{align*}
\mathcal P&(\emptyset) = \{ \emptyset \}\\
\mathcal P&(\{ a \}) = \bigl\{ \emptyset, \{ a \} \bigr\}\\
\mathcal P&(\{ a, b \}) = \bigl\{ \emptyset, \{ a \}, \{ b \}, \{ a, b \} \bigr\}\\
\end{align*}

\end{document}